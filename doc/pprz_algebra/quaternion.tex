\section{Quaternion}
\mynote{I hate the naming convention for the real part.}
\subsection{Definition}
The values are called
\begin{equation}
\quat{} = q_i + \mathrm{i} q_x + \mathrm{j} q_y + \mathrm{k} q_z
 = \begin{pmatrix} q_i \\q_x\\q_y\\q_z\end{pmatrix}
\end{equation}
It is available for the following simple types:\\
\begin{tabular}{c|c}
type		& struct		\\ \hline
int32\_t	& Int32Quat		\\
float		& FloatQuat		\\
double		& DoubleQuat		
\end{tabular}




\subsection{= Assigning}
\subsubsection*{$\quat{} = $ Identity}
Sets a quaternion to the identity rotation (no rotation).
\begin{equation}
\quat{} = 1 = \begin{pmatrix}1 \\ 0 \\0\\0\end{pmatrix}
\end{equation}
\inHfile{INT32\_QUAT\_ZERO(q)}{pprz\_algebra\_int}
\inHfile{FLOAT\_QUAT\_ZERO(q)}{pprz\_algebra\_float}

\subsubsection*{$\quat{} = \transp{(\quat i,\quat x,\quat y,\quat z)}$}
\textit{i} is the real part of the quaternion.
\begin{equation}
 \quat a = \begin{pmatrix}i\\ x \\ y \\ z \end{pmatrix}
\end{equation}
\inHfile{QUAT\_ASSIGN(q, i, x, y, z)}{pprz\_algebra}

\subsubsection*{$\quat{o} = \quat{i}$}
\begin{equation}
\quat o = \quat i
\end{equation}
\inHfile{QUAT\_COPY(qo, qi)}{pprz\_algebra}
\inHfile{FLOAT\_QUAT\_COPY(qo, qi)}{pprz\_algebra\_float}

\subsubsection*{$\quat{b} = - \quat{a}$}
\begin{equation}
\quat b = - \quat a
\end{equation}
\inHfile{QUAT\_EXPLEMENTARY(b, a)}{pprz\_algebra}
\inHfile{FLOAT\_QUAT\_EXPLEMENTARY(b, a)}{pprz\_algebra\_float}
\mynote{Naming the other way round?}



\subsection{+ Addition}
\subsubsection*{$\quat{o} += \quat{i}$}
\begin{equation}
\quat o = \quat o + \quat i
\end{equation}
\inHfile{QUAT\_ADD(qo, qi)}{pprz\_algebra}
\inHfile{FLOAT\_QUAT\_ADD(qo, qi)}{pprz\_algebra\_float}
\mynote{No SUM function?}



\subsection{- Subtraction}
\subsubsection*{$\quat{c} = \quat{a} - \quat{b}$}
\begin{equation}
\quat c = \quat a - \quat b
\end{equation}
\inHfile{QUAT\_DIFF(qc, qa, qb)}{pprz\_algebra}
\mynote{no SUB function?}



\subsection{$\multiplication$ Multiplication}
\mynote{FLOAT\_QUAT\_ROTATE\_FRAME is stil missing. The function seems useless to me.}
\subsubsection*{$\quat{o} = s \multiplication \quat{i}$ With a scalar}
\begin{equation}
	\quat{o} = s \multiplication \quat{i}
\end{equation}
\inHfile{QUAT\_SMUL(vo, vi, s)}{pprz\_algebra}
\inHfile{FLOAT\_QUAT\_SMUL(vo, vi, s)}{pprz\_algebra\_float}

\subsubsection*{$\quat{a2c} = \quat{b2c} \quatprod \quat{a2b}$ With a quaternion (composition)}
Returns the multiplication/composition of two quaternions.
\begin{equation}
\quat{a2c} = \quat{b2c} \quatprod \quat{a2b}
\end{equation}
\begin{equation}
\quat{a2c} = \begin{pmatrix}
\quat{b2c,i} & -\quat{b2c,x} & -\quat{b2c,y} & -\quat{b2c,z} \\
\quat{b2c,x} &  \quat{b2c,i} & -\quat{b2c,z} &  \quat{b2c,y} \\
\quat{b2c,y} &  \quat{b2c,z} &  \quat{b2c,i} & -\quat{b2c,x} \\
\quat{b2c,z} & -\quat{b2c,y} &  \quat{b2c,x} &  \quat{b2c,i}
\end{pmatrix}\multiplication \begin{pmatrix}
\quat{a2b,i}\\\quat{a2b,x}\\\quat{a2b,y}\\\quat{a2b,z}
\end{pmatrix}
\end{equation}
\inHfile{INT32\_QUAT\_COMP(a2c, a2b, b2c) }{pprz\_algebra\_int}
\inHfile{FLOAT\_QUAT\_COMP(a2c, a2b, b2c) }{pprz\_algebra\_float}
\inHfile{FLOAT\_QUAT\_MULT(a2c, a2b, b2c) }{pprz\_algebra\_float}
Also available with inversions/conjugations (please note, that a inversion and a conjugation is the same for a unit quaternion):
\begin{equation}
\quat{a2b} = \comp{\quat{b2c}} \quatprod \quat{a2c}
\end{equation}
\inHfile{INT32\_QUAT\_COMP\_INV(a2b, a2c, b2c) }{pprz\_algebra\_int}
\inHfile{FLOAT\_QUAT\_COMP\_INV(a2b, a2c, b2c) }{pprz\_algebra\_float}
\begin{equation}
\quat{b2c} = \quat{a2c} \quatprod \comp{\quat{a2b}}
\end{equation}
\inHfile{INT32\_QUAT\_INV\_COMP(b2c, a2b, a2c)}{pprz\_algebra\_int}
\inHfile{FLOAT\_QUAT\_INV\_COMP(b2c, a2b, a2c)}{pprz\_algebra\_float}
\emph{Note}
Please note that due to the fact that it's done very often, the functions above are also available with Normalisation:
\inHfile{FLOAT\_QUAT\_COMP\_INV\_NORM\_SHORTEST(a2b, a2c, b2c)}{pprz\_algebra\_float}
\inHfile{FLOAT\_QUAT\_INV\_COMP\_NORM\_SHORTEST(b2c, a2b, a2c)}{pprz\_algebra\_float}

\mynote{no Division?}

\subsection{$\comp{}$ Complementary}
\begin{equation}
\quat o = \comp{\quat i}
\end{equation}
\inHfile{QUAT\_INVERT(qo, qi)}{pprz\_algebra}
\inHfile{INT32\_QUAT\_INVERT(qo, qi)}{pprz\_algebra\_int}
\inHfile{FLOAT\_QUAT\_INVERT(qo, qi)}{pprz\_algebra\_float}



\subsection{Transformation from Quaternions}
\subsubsection*{to a rotational matrix}
The most common definition for this transformation is
\begin{equation}
\mat R_m = \begin{pmatrix}
1-2(q_y^2 + q_z^2)		& 2(q_xq_y-q_iq_z)		& 2(q_xq_z + q_iq_y) \\
2(q_xq_y + q_iq_z)		& 1-2(q_x^2 + q_z^2)	& 2(q_yq_z - q_iq_x) \\
2(q_xq_z - q_iq_y)		& 2(q_yq_z+q_iq_x)		& 1-2(q_x^2 + q_y^2)	
\end{pmatrix}.
\end{equation}

\inHfile{INT32\_RMAT\_OF\_QUAT(rm, q)}{pprz\_algebra\_int}
\inHfile{FLOAT\_RMAT\_OF\_QUAT(rm, q)}{pprz\_algebra\_float}
\mynote{I called the quicker function "INT32\_RMAT\_OF\_QUAT\_QUICKER"}



\subsubsection*{to euler angles}
This is done by constructing a rotational matrix out of a quaternion (note: not all elements need to be generated), 
\begin{equation}
\mat R_m = \begin{pmatrix}
1-2(q_y^2 + q_z^2)		& 2(q_xq_y-q_iq_z)		& 2(q_xq_z + q_iq_y) \\
						& 						& 2(q_yq_z - q_iq_x) \\
						& 						& 1-2(q_x^2 + q_y^2)	
\end{pmatrix},
\end{equation}
which is equivalent to a rotational matrix, that is constructed from euler angles
\begin{equation}
\mat R_m = \begin{pmatrix}
cos(\Pitch)cos(\Yaw)	& cos(\Pitch)sin(\Yaw)	& -sin(\Pitch)			\\
						& 						& sin(\Roll)cos(\Pitch)	\\
						& 						& cos(\Roll)cos(\Pitch)
\end{pmatrix}.
\end{equation}
The euler angles are then
\begin{equation}
\eu e = \begin{pmatrix}\Roll \\ \Pitch \\ \Yaw \end{pmatrix} = 
\begin{pmatrix}
\arctan2(r_{23}, r_{33}) \\
-\arcsin(r_{13}) \\
\arctan2(r_{12}, r_{11})
\end{pmatrix}
\end{equation}
\inHfile{INT32\_EULERS\_OF\_QUAT(e, q)}{pprz\_algebra\_int}
\inHfile{FLOAT\_EULERS\_OF\_QUAT(e, q)}{pprz\_algebra\_float}
\inHfile{DOUBLE\_EULERS\_OF\_QUAT(e, q)}{pprz\_algebra\_float}




\subsection{Transformation to Quaternions}
\subsubsection*{from an axis and an angle}
A quaternion can be easily constructed from an axis $\vect u_v$ and an angle $\alpha $ using
\begin{equation}
\quat {} = \begin{pmatrix}
\cos \left( \tfrac \alpha 2 \right) \\
\sin \left( \tfrac \alpha 2 \right) \vect u_v
\end{pmatrix} = \begin{pmatrix}
\cos \left( \tfrac \alpha 2 \right) \\
\sin \left( \tfrac \alpha 2 \right) u_x \\
\sin \left( \tfrac \alpha 2 \right) u_y \\
\sin \left( \tfrac \alpha 2 \right) u_z
\end{pmatrix}
\end{equation}
\inHfile{FLOAT\_QUAT\_OF\_AXIS\_ANGLE(q, uv, an)}{pprz\_algebra\_float}

\subsubsection*{from euler angles}
The transformation is given by
\begin{equation}
\quat{} = [\cos \tfrac{\Yaw}{2} + \mathbf{k} \sin \tfrac{\Yaw}{2}][\cos \tfrac{\Pitch}{2} + \mathbf{j} \sin \tfrac{\Pitch}{2}][\cos \tfrac{\Roll}{2} + \mathbf{i} \sin \tfrac{\Roll}{2}]
\end{equation}
In matrix notation:
\begin{equation}
\quat = \begin{pmatrix}
\cos \tfrac{\Roll}{2} \cos \tfrac{\Pitch}{2} \cos \tfrac{\Yaw}{2} + \sin \tfrac{\Roll}{2} \sin \tfrac{\Pitch}{2} \sin \tfrac{\Yaw}{2} \\
\sin \tfrac{\Roll}{2} \cos \tfrac{\Pitch}{2} \cos \tfrac{\Yaw}{2} - \cos \tfrac{\Roll}{2} \sin \tfrac{\Pitch}{2} \sin \tfrac{\Yaw}{2} \\
\cos \tfrac{\Roll}{2} \sin \tfrac{\Pitch}{2} \cos \tfrac{\Yaw}{2} + \sin \tfrac{\Roll}{2} \cos \tfrac{\Pitch}{2} \sin \tfrac{\Yaw}{2} \\
\cos \tfrac{\Roll}{2} \cos \tfrac{\Pitch}{2} \sin \tfrac{\Yaw}{2} - \sin \tfrac{\Roll}{2} \cos \tfrac{\Pitch}{2} \sin \tfrac{\Yaw}{2}
\end{pmatrix}
\end{equation}
\inHfile{INT32\_QUAT\_OF\_EULERS(q, e)}{pprz\_algebra\_int}
\inHfile{FLOAT\_QUAT\_OF\_EULERS(q, e)}{pprz\_algebra\_float}
\inHfile{DOUBLE\_QUAT\_OF\_EULERS(q, e)}{pprz\_algebra\_double}


\subsubsection*{from a rotational matrix}
Since the construction of a matrix from a quaternion is known
\begin{equation}
\mat R_m = \begin{pmatrix}
1-2(q_y^2 + q_z^2)		& 2(q_xq_y-q_iq_z)		& 2(q_xq_z + q_iq_y) \\
2(q_xq_y + q_iq_z)		& 1-2(q_x^2 + q_z^2)	& 2(q_yq_z - q_iq_x) \\
2(q_xq_z - q_iq_y)		& 2(q_yq_z+q_iq_x)		& 1-2(q_x^2 + q_y^2)	
\end{pmatrix},
\end{equation}
the extraction of a quaternion is done vice versa. But there are obviously many opportunities to extract the quaternion. They differ in the way which element of the quaternion is extracted from the diaognal elements $r_{11}$, $r_{22}$ and $r_{33}$  of the matrix.
\begin{equation}
1 = q_i^2+q_x^2+q_y^2+q_z^2
\end{equation}
\textbf{First case}
\begin{eqnarray}
\zeta = \sqrt{1 + (r_{11}+r_{22}+r_{33})}  =  \sqrt{1 + (3q_i^2 -q_x^2 -q_y^2 -q_z^2)} = \sqrt{4 q_i^2} \\
q_i = \tfrac 1 2 \zeta \\
q_x =  \tfrac 1 {2 \zeta} (r_{23}-r_{32}) \\
q_y =  \tfrac 1 {2 \zeta} (r_{31}-r_{13}) \\
q_z =  \tfrac 1 {2 \zeta} (r_{12}-r_{21})
\end{eqnarray}
\textbf{Second case}
\begin{eqnarray}
\zeta = \sqrt{1 + (r_{11}-r_{22}-r_{33})}  =  \sqrt{1 + (-q_i^2+3q_x^2 -q_y^2 -q_z^2)} = \sqrt{4 q_x^2} \\
q_i =  \tfrac 1 {2 \zeta} (r_{23}-r_{32}) \\
q_x = \tfrac 1 2 \zeta \\
q_y =  \tfrac 1 {2 \zeta} (r_{12}+r_{21}) \\
q_z =  \tfrac 1 {2 \zeta} (r_{31}+r_{13})
\end{eqnarray}
\textbf{Third case}
\begin{eqnarray}
\zeta = \sqrt{1 + (-r_{11}+r_{22}-r_{33})}  =  \sqrt{1 + (-q_i^2 -q_x^2+3q_y^2 -q_z^2)} = \sqrt{4 q_y^2} \\
q_i =  \tfrac 1 {2 \zeta} (r_{31}-r_{13}) \\
q_x =  \tfrac 1 {2 \zeta} (r_{12}+r_{21}) \\
q_y = \tfrac 1 2 \zeta \\
q_z =  \tfrac 1 {2 \zeta} (r_{23}+r_{32})
\end{eqnarray}
\textbf{Fourth case}
\begin{eqnarray}
\zeta = \sqrt{1 + (-r_{11}-r_{22}+r_{33})}  =  \sqrt{1 + (-q_i^2 -q_x^2 -q_y^2+3q_z^2)} = \sqrt{4 q_z^2} \\
q_i =  \tfrac 1 {2 \zeta} (r_{12}-r_{21}) \\
q_x =  \tfrac 1 {2 \zeta} (r_{31}+r_{13}) \\
q_y =  \tfrac 1 {2 \zeta} (r_{23}+r_{32}) \\
q_z = \tfrac 1 2 \zeta
\end{eqnarray}
All are mathematicaly equivalent but numerically different. To avoid complex numbers and singularities the case with the biggest $\zeta$ should be choosen. 
\inHfile{INT32\_QUAT\_OF\_RMAT(q, r)}{pprz\_algebra\_int}
\inHfile{FLOAT\_QUAT\_OF\_RMAT(q, r)}{pprz\_algebra\_float}

\subsubsection*{from measured rates}
This function computes the differential quaternion of measured rates after an amount of time. \\
Let $ \ra{} $ be the measured rates, then $\norm {\ra {}}$ represents the absolute value of rates. Therefore,
\begin{equation}
\Delta \alpha = \norm {\ra {}} \multiplication \Delta t
\end{equation}
is the rotational angle. The (normalized) axis of the rotation is then
\begin{equation}
\vect v = \frac{\ra{}}{\norm {\ra {}}}.
\end{equation}
The construction of a quaternion from an axis and an angle is
\begin{equation}
\quat{} = \begin{pmatrix}
\cos \tfrac \alpha 2 \\
\vect v \sin \tfrac \alpha 2
\end{pmatrix},
\end{equation}
so that the resulting quaternion of measured rates becomes
\begin{equation}
\quat{} = \begin{pmatrix}
\cos \tfrac{\norm {\ra {}} \multiplication \Delta t} 2 \\
 \frac{\ra{}}{\norm {\ra {}}} \sin \tfrac{\norm {\ra {}} \multiplication \Delta t} 2
\end{pmatrix}.
\end{equation}
\inHfile{FLOAT\_QUAT\_DIFFERENTIAL(q\_out, w, dt)}{pprz\_algebra\_float}



\subsection{Other}
\subsubsection*{$\norm{\quat{}}$ Norm}
Returns the 2-norm of a quaternion
\begin{equation}
n = \norm{\quat{}} = \sqrt{\quat{}\comp{\quat{}}} = \sqrt{\quat i^2 + \quat x^2 + \quat y^2 + \quat z^2}
\end{equation}
\inHfile{INT32\_QUAT\_NORM(n, q)}{pprz\_algebra\_int}
\inHfile{FLOAT\_QUAT\_NORM(n, q)}{pprz\_algebra\_float}
It is also possible to directly normalise the quaternion
\begin{equation}
\quat{} := \frac{\quat{}}{\norm{\quat{}}}
\end{equation}
\inHfile{INT32\_QUAT\_NORMALIZE(q)}{pprz\_algebra\_int}
\inHfile{FLOAT\_QUAT\_NORMALIZE(q)}{pprz\_algebra\_float}

\subsection*{Making the real value positive}
It is possible to invert the quaternion if its real value is negative
\begin{equation}
\quat{} = \left\lbrace \begin{matrix}
\quat{} \quad \quat i>0 \\
-\quat{} \quad \quat i<0
\end{matrix} \right.
\end{equation}
\inHfile{INT32\_QUAT\_WRAP\_SHORTEST(q)}{pprz\_algebra\_int}
\inHfile{FLOAT\_QUAT\_WRAP\_SHORTEST(q)}{pprz\_algebra\_float}

\subsection*{Derivative}
Calculates the derivative of a quaternion using the rates. The resulting quaternion still needs to be normalized
\begin{equation}
\dot{\quat{}} = -\tfrac 1 2 \mat \Omega(\ra{}) \quatprod \quat{}
\end{equation}
\begin{equation}
\dot{\quat{}} = -\tfrac 1 2 \begin{pmatrix} 
  0		&  \ra p &  \ra q &  \ra r	\\
-\ra p	&	0	 & -\ra r &  \ra q	\\
-\ra q	&  \ra r &	0	  & -\ra p	\\
-\ra r	& -\ra q &  \ra p &		0
\end{pmatrix}
\multiplication \quat{}
\end{equation}
\inHfile{FLOAT\_QUAT\_DERIVATIVE(qd, r, q)}{pprz\_algebra\_float}
You can also use a method, which slightly normalizes the quaternion by itself. The intention is that you calculate a quaternion, which represents the difference to a unit quaternion
\begin{eqnarray}
\Delta n = \norm{\norm{\quat{}}}_2-1 \\
\Delta \quat {} = \Delta n \multiplication \quat{}.
\end{eqnarray}
Now you substract this difference from the result
\begin{eqnarray}
\dot{\quat{}} = -\tfrac 1 2 \mat \Omega(\ra{}) \quatprod \quat{} - \Delta \quat {} \\
\dot{\quat{}} = -\tfrac 1 2 \mat \Omega(\ra{}) \quatprod \quat{} - \Delta n \multiplication \quat{} \\
\dot{\quat{}} = -\tfrac 1 2 \left( 2 \Delta n \eye + \mat \Omega(\ra{}) \right) \quatprod \quat{}
\end{eqnarray}
leading to
\begin{equation}
\dot{\quat{}} = -\tfrac 1 2 \begin{pmatrix} 
2 \Delta n&  \ra p &  \ra q &  \ra r	\\
-\ra p	&2 \Delta n& -\ra r &  \ra q	\\
-\ra q	&  \ra r &2 \Delta n & -\ra p	\\
-\ra r	& -\ra q &  \ra p &2 \Delta n
\end{pmatrix}
\multiplication \quat{}
\end{equation}
\inHfile{FLOAT\_QUAT\_DERIVATIVE\_LAGRANGE(qd, r, q)}{pprz\_algebra\_float}
